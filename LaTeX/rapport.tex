\documentclass{article}

\usepackage[utf8]{inputenc}
\usepackage[T1]{fontenc}
\usepackage[french]{babel}
 
\usepackage[margin=1in]{geometry} 
\usepackage{amsmath,amsthm,amssymb}
\usepackage{enumitem}
\usepackage{color}
\usepackage{listings}
\usepackage{graphicx}
\usepackage{here}

\definecolor{gray}{rgb}{0.5,0.5,0.5}
\definecolor{mygreen}{rgb}{0,0.6,0}
\lstset{
language=R,
basicstyle=\scriptsize\ttfamily,
commentstyle=\ttfamily\color{mygreen},
numbers=left,
numberstyle=\ttfamily\color{gray}\footnotesize,
stepnumber=1,
numbersep=5pt,
backgroundcolor=\color{white},
showspaces=false,
showstringspaces=false,
showtabs=false,
frame=single,
tabsize=2,
captionpos=b,
breaklines=true,
breakatwhitespace=false,
title=\lstname,
escapeinside={},
keywordstyle=\color{blue},
morekeywords={}
}
 
\newcommand{\N}{\mathbb{N}}
\newcommand{\Z}{\mathbb{Z}}
\newcommand{\R}{\mathbb{R}}
\newcommand{\E}{\mathbb{E}}
\newcommand{\deriv}{\mathrm{d}}

\addto\captionsfrench{%
  \renewcommand{\figurename}{Graphique}%
}

\title{ENSAE - 3A
\\ Chaînes de Markov cachées et méthodes de Monte Carlo séquentielles
\\ Application à l'astrophysique}
\author{Bruno ABILOU - Anouar BARCHID - Samuel GIVOIS 
    \\ Enseignant : M. Nicolas CHOPIN}
\date{13 janvier 2017}

%%%%%%%%%%%%%%%%%%%%%%%%%%%%%%%%%%%%%%%%%%%%%%%%%%
%%%%% DEBUT DU DOCUMENT
%%%%%%%%%%%%%%%%%%%%%%%%%%%%%%%%%%%%%%%%%%%%%%%%%%
\begin{document}

\maketitle

%%%%%%%%%%%%%%%%%%%%%%%%%%%%%%%%%%%%%%%%%%%%%%%%%%
%%%%% 1 - Description du modele
\section{Description du modèle}

Ce travail repose sur un modèle proposé par Yu et Meng en 2011 \cite{yu2011center}. Dans cet article, les auteurs proposent une stratégie améliorer l'efficacité d'algorithmes de Monte-Carlo par chaînes de Markov (MCMC). Ils l'appliquent notamment à un modèle hiérarchique bayésien du nombre de photons.

Le modèle est défini par quatre paramètres : $\beta$ (de dimension 2), $\delta$ et $\rho$. Ceux-ci caractérisent la loi supposée du nombre de photons observé au cours du temps $Y_t$, qui dépend aussi d'un processus auto-régressif stationnaire latent $\xi_t$. Formellement, le modèle est le suivant :
\[
\begin{array}{c}
Y_t | (\xi_t, \beta) \sim \mathcal{P}(d_t e^{X_t \beta + \xi_t}) \\
\xi_t | (\xi_1, ..., \xi_{t-1}, \rho, \delta) \sim \mathcal{N}(\rho\xi_{t-1}, \delta^2) \ \ \ (t>1) \\
\xi_1 \sim \mathcal{N}(0, \frac{\delta^2}{1-\rho^2})
\end{array}
\]
où $d_t$ est une suite connue, et $X_t$ correspond au couple $(1, \frac{t}{T})$.

Pour les auteurs, la question d'intérêt sur ce modèle est celle de savoir si le nombre dépend du temps écoulé, \textit{i.e} déterminer si la deuxième composante de $\beta$ est différente de $0$. Ils proposent pour se faire une estimation bayésienne du paramètre $\theta = (\beta, \delta, \rho)$. Celle-ci est fondée sur un échantillonnage selon la loi a posteriori de $\theta$ sachant les données, pour une loi a priori constante $p(\beta, \delta, \rho) \propto 1$. Leur solution repose sur des MCMC, et consiste en une amélioration du \og standard Gibbs sampler\fg. 

On note $l(y_1, ..., y_t | \theta)$ la vraisemblance du modèle et $q$ la densité de $\theta$ sachant les données observées. D'après la formule de Bayes, on a :
\[
q(\theta | y_1, ..., y_t) = \frac{l(y_1, ..., y_t | \theta)p(\theta)}{p(y_1, ..., y_t)}
\]
Pour échantillonner selon la loi a posteriori de $\theta$, il est donc nécessaire de savoir calculer la vraisemblance du modèle (au moins à une constante près). Or, ce n'est pas possible ici car cette vraisemblance dépend du processus $(\xi_t)$ qui n'est pas observable. On va donc chercher à contourner cette difficulté en utilisant des filtres particulaires et des algorithmes PMCMC (\cite{andrieu2010particle} et \cite{holenstein2009particle}).

%%%%%%%%%%%%%%%%%%%%%%%%%%%%%%%%%%%%%%%%%%%%%%%%%%
%%%%% * - Bibliographie
\renewcommand\bibname{Références}
\nocite{*}
\bibliographystyle{plain}
\bibliography{biblio}

\end{document}
